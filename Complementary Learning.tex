% Options for packages loaded elsewhere
\PassOptionsToPackage{unicode}{hyperref}
\PassOptionsToPackage{hyphens}{url}
%
\documentclass[
]{article}
\usepackage{lmodern}
\usepackage{amssymb,amsmath}
\usepackage{ifxetex,ifluatex}
\ifnum 0\ifxetex 1\fi\ifluatex 1\fi=0 % if pdftex
  \usepackage[T1]{fontenc}
  \usepackage[utf8]{inputenc}
  \usepackage{textcomp} % provide euro and other symbols
\else % if luatex or xetex
  \usepackage{unicode-math}
  \defaultfontfeatures{Scale=MatchLowercase}
  \defaultfontfeatures[\rmfamily]{Ligatures=TeX,Scale=1}
\fi
% Use upquote if available, for straight quotes in verbatim environments
\IfFileExists{upquote.sty}{\usepackage{upquote}}{}
\IfFileExists{microtype.sty}{% use microtype if available
  \usepackage[]{microtype}
  \UseMicrotypeSet[protrusion]{basicmath} % disable protrusion for tt fonts
}{}
\makeatletter
\@ifundefined{KOMAClassName}{% if non-KOMA class
  \IfFileExists{parskip.sty}{%
    \usepackage{parskip}
  }{% else
    \setlength{\parindent}{0pt}
    \setlength{\parskip}{6pt plus 2pt minus 1pt}}
}{% if KOMA class
  \KOMAoptions{parskip=half}}
\makeatother
\usepackage{xcolor}
\IfFileExists{xurl.sty}{\usepackage{xurl}}{} % add URL line breaks if available
\IfFileExists{bookmark.sty}{\usepackage{bookmark}}{\usepackage{hyperref}}
\hypersetup{
  hidelinks,
  pdfcreator={LaTeX via pandoc}}
\urlstyle{same} % disable monospaced font for URLs
\setlength{\emergencystretch}{3em} % prevent overfull lines
\providecommand{\tightlist}{%
  \setlength{\itemsep}{0pt}\setlength{\parskip}{0pt}}
\setcounter{secnumdepth}{-\maxdimen} % remove section numbering

\author{}
\date{}

\begin{document}

\hypertarget{header-n0}{%
\subsection{Complementary Learning}\label{header-n0}}

\textbf{Standard Deviation}

\begin{itemize}
\item
  measure of how spread out numbers are
\end{itemize}

\[\sigma = \sqrt{{\frac{1}{N}}\sum^N_{i=1}(x_i - \mu)^2}\]

\begin{itemize}
\item
  \(\mu\) is the mean or average of the numbers
\item
  subtract the mean from each number and square the result
\item
  find the mean of those squared differences
\item
  take the square root
\end{itemize}

\textbf{Z-Score / Z-Test}

\begin{itemize}
\item
  finds how far from the mean a data point is using population standard
  deviation

  \[Z = \frac{x-\mu}{\sigma}\]

  \begin{itemize}
  \item
    where \(x\) is the tested data point
  \item
    \(\sigma\) is the population standard deviation
  \item
    \(\mu\) is the population mean
  \item
    \(Z\) is the number of standard deviations the point is from the
    mean
  \end{itemize}
\item
  for example, a z-score of 1.6 means that the result is 1.6 standard
  deviations away from the mean
\item
  use a z-score to calculate critical value. For an alpha level of 5\%
  or more extreme, the z-score must be greater than 1.645 because the
  area on the left side of a standard distribution is 95\% of the
  distribution.
\item
  p-values and critical values of z can be used interchangeably
\end{itemize}

\textbf{T-Statistic}

\begin{itemize}
\item
  Use the t-score when you don't know the population standard deviation
  or have a small sample size

  \[T = \frac{\bar{x}-\mu_0}{\sigma} =\frac{\bar{x}-\mu_0}{s/\sqrt{n}}\]

  \begin{itemize}
  \item
    Where \(\bar{x}\) is the sample mean
  \item
    \(\mu\) is the population mean
  \item
    \(n\) is the sample size
  \item
    \(s\) is the sample standard deviation
  \end{itemize}
\item
  can be used to find the difference between 2 means
\item
  similar to the z-score

  \begin{itemize}
  \item
    find a cut off point
  \item
    find the t-Score and compare
  \end{itemize}
\item
  Has thicker tail because the standard deviation of the population is
  being tested

  \[t = \frac{\hat\beta_1 - 0}{SE(\hat\beta_1)}\]
\end{itemize}

In null hypothesis testing for coefficients, the standard error is a
estimate of the standard deviation. This t-statistic is compared with
values in the student's t-distribution to determine the p-value.
Student's t-distribution describes how the mean of a sample with a \(n\)
number of observations is expected to behave. The p-value is the
probability of seeing a number as extreme or more than the t-statistic
in a collection of random data where the variable has no effect.

\textbf{P-Value}

\begin{itemize}
\item
  probability that a point is as extreme or more assuming the null
  hypothesis
\item
  With a p-value of 0.05 or 5\% there is only a 5\% chance that this
  coefficient would have occurred in a random distribution

  \begin{itemize}
  \item
    depending on the cutoff, might lead to rejecting the null hypothesis
  \end{itemize}
\end{itemize}

\textbf{F-Statistic}

\begin{align}
F &= \frac{\text{explained variation}/(p-1)}{\text{unexaplained variation}/(n-p)}\\
F &= \frac{R^2/(p-1)}{(1-R^2)/(n-p)}
\end{align}

\(R^2\) measures the relationship between the predictors and response,
not whether the relationship is statistically significant. The
f-statistic judges whether the relationship between the predictors and
the response is statistically significant.

\textbf{Confidence Intervals}

\[\bar{x} \pm z\sigma_{\bar{X}}\]

\begin{itemize}
\item
  \(\bar{x}\) is the sample mean
\item
  \(z\) is the z-score or the number of standard deviations based on the
  confidence level
\item
  \(\sigma_{\bar{X}}\) is the standard error of the mean
\end{itemize}

\[\hat{\beta_1} \pm 2 \cdot SE(\hat{\beta_1})\]

represents the confidence interval of \(\hat{\beta}\) to 95\%.

\end{document}
